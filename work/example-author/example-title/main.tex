\Title{タイトル}{作者名}

シルバーグレーの長髪は、紫のインナーカラーをそっと隠してその毛先だけがくるりと内側にカールしている。前髪は短く切り揃えられていて、PR‐B世代\footnote{匿名掲示板ではたくあん世代と呼ばれている。}に特徴的な太めの眉がよく見える。やっぱり可愛い。

「こんにちは。リリといいます」

サービスロイドっぽい甘めの顔が私を見上げて、値踏みするような目を向ける私を気にも留めないふうに、うやうやしく一礼した。声帯型発声器ではなく喉のスピーカーから鳴る声も、この世代の大きな特徴である。

\sectionbreak

「GPSも使えなくなってるみたい。どうしよう……」

さっきまでお店との連絡に使っていたはずのタブレットは、もう使い物にならないらしい。リリが画面をなぞる手に焦りがにじむ。さっきとキャラが違うけど、これが素のリリってことなんだろうか。

インターネットに接続できなくて状況が分からないというので、スマホの背面にタッチしたリリの青く光る指先に、\ruby{近接通信}{ニアバイ}で無線LANのパスワードを渡した。

「けど、なんで摘発されたの?」

しかも、用意周到に証拠隠滅の準備まで。まるで、初めから捕まることを見越していたかのようだ。

「知らないわ。もしかしたら、倫理規定違反かも。少なくとも、時間外使用はもはや言い逃れできないレベルだったから」

倫理規定\――サービスロイド使用倫理規定は、ピンクキャブのようにセクサロイドを扱うデリヘルが最も気を遣う規則だ。サービスロイドを守るという建て付けで、一定の条件を満たすあらゆる自律型ヒューマノイドに対して、一律に\bou{人間風の}強い保護を与えることを定めている。しかし、メンテナンスやパーツ交換が容易で疲れることもないロボットを守るという視点では、理不尽で無意味な規制というほかない。

\sectionbreak

リリはあの日から、何も話さなくなった。彼女には飲食も排泄も入浴も必要ないから、放っておくとずっと部屋の隅に座ったままだ。膝を抱えて座っていてもインジケータは動作したままなので、まるで隣に置かれたルーターと会話しているようにも見える。

彼女が身体を動かす唯一のタイミングは、バッテリーが切れる直前だけ。サービスロイドは自らを充電できないように制限されている\footnote{ただし、鍵のような凹凸パターンが施された固定型の充電アダプタを取り付けることで、プラグを設置した場所でのみ自ら充電できるようになる。通常は、腰部の拡張用空間(リリは既にセックスメーターが入っている)に鍵穴のようなコネクタを取り付ける。}ので、「サナ、お願い」と私に充電プラグを挿すように頼まなければらないのだ。

私は、リリに何と言うべきか分からなかった。酒に酔ってやったことだから仕方ない、なんて言うつもりはなかったけど、どうしてあんなことをしてしまったのか、私にも分からない。ごめんねリリ、でも過去なんて気にしないで、私には隠さなくていいから……何を言ったとしても、彼女は悔しそうに私を睨みつけるだろう。

あるいは、リリだって苦しんでいるんだから、彼女の言う通り強力な鍵で好き勝手にストレージを操作して、記憶ごと消せばいいんじゃないか\?サービスロイドの忘れる権利は倫理規定でも保障されている。私の手できれいさっぱり無かったことにしてあげたほうが、彼女だって幸せなんじゃないか……と、どこからともなく浮かぶ身勝手な考えを振り払うように首を振った。

その一線は、超えちゃだめだ。そんなことをしたら、私とリリは人間と道具の関係に成り下がってしまう。
